\documentclass[review]{elsarticle}

\usepackage{lineno,hyperref}
\modulolinenumbers[5]

\journal{Journal of \LaTeX\ Templates}

%%%%%%%%%%%%%%%%%%%%%%%
%% Elsevier bibliography styles
%%%%%%%%%%%%%%%%%%%%%%%
%% To change the style, put a % in front of the second line of the current style and
%% remove the % from the second line of the style you would like to use.
%%%%%%%%%%%%%%%%%%%%%%%

%% Numbered
%\bibliographystyle{model1-num-names}

%% Numbered without titles
%\bibliographystyle{model1a-num-names}

%% Harvard
%\bibliographystyle{model2-names.bst}\biboptions{authoryear}

%% Vancouver numbered
%\usepackage{numcompress}\bibliographystyle{model3-num-names}

%% Vancouver name/year
%\usepackage{numcompress}\bibliographystyle{model4-names}\biboptions{authoryear}

%% APA style
%\bibliographystyle{model5-names}\biboptions{authoryear}

%% AMA style
%\usepackage{numcompress}\bibliographystyle{model6-num-names}

%% `Elsevier LaTeX' style
\bibliographystyle{./bib/elsarticle-num}
%%%%%%%%%%%%%%%%%%%%%%%

\begin{document}

\begin{frontmatter}

\title{Quantum Lattice Model Solver: $\mathcal{H} \Phi$}
%\tnotetext[mytitlenote]{Fully documented templates are available in the elsarticle package on \href{http://www.ctan.org/tex-archive/macros/latex/contrib/elsarticle}{CTAN}.}

%% Group authors per affiliation:
\author{Youhei Yamaji}
\address{Quantum-Phase Electronics Center (QPEC), The University of Tokyo, Bunkyo-ku, Tokyo, 113-8656, Japan}
\ead{yamaji@ap.t.u-tokyo.ac.jp}
%\fntext[myfootnote]{Since 1880.}

%% or include affiliations in footnotes:
%\author[myaddress]{Quantum-Phase Electronics Center (QPEC), The University of Tokyo}
%\ead{yamaji@ap.t.u-tokyo.ac.jp}

%\author[mysecondaryaddress]{Global Customer Service\corref{mycorrespondingauthor}}
%\cortext[mycorrespondingauthor]{Corresponding author}
%\ead{support@elsevier.com}

%\address[mymainaddress]{1600 John F Kennedy Boulevard, Philadelphia}
%\address[mysecondaryaddress]{360 Park Avenue South, New York}

\begin{abstract}
This template helps you to create a properly formatted \LaTeX\ manuscript.
\end{abstract}

\begin{keyword}
\texttt{elsarticle.cls}\sep \LaTeX\sep Elsevier \sep template
\MSC[2010] 00-01\sep  99-00
\end{keyword}

\end{frontmatter}

\linenumbers

\newcommand{\HPhi}{ \mathcal{H} \Phi}
\section{Introduction}

Comparison between experimental observation and theoretical analysis is a crucial step
in condensed-matter physics research. Temperature dependence of specific heat and
magnetic susceptibility, for example, have been studied to extract nature of low energy
excitations of and magnetic interactions among electrons, respectively, through comparison
with theories such as Landau's Fermi liquid theory and Curie-Weiss law.

For the flexible and quantitative comparison with experimental data, an exact diagonalization
approach~\cite{Dagotto} is one of the most reliable numerical tools without any approximation or
inspiration of genius. For last few decades, a numerical diagonalization package for quantum
spin hamiltonians, TITPACK developed by Prof. Hidetoshi Nishimori in Tokyo Institute of Technology,
has been widely used in the condensed-matter physics community. Nevertheless, limitation of
computational resources had hindered the non-expert users from applying the package to
quantum systems with large number of electrons or spins.

In contrast, recent and rapid development of parallel computing infrastructure opens up new
avenues for user-friendly larger scale diagonalizations up to 18 site Hubbard clusters
or 36 $S$=$1/2$ quantum spins. In addition, recent advances in quantum statistical mechanics ~\cite{Imada1986,FTLanczos,Hams,Sugiura2012}
enable us to calculate finite temperature properties of quantum many-body systems
with computational costs similar to calculations of ground state properties,
which also enables us to compare theoretical results for temperature dependence
of, for example, specific heat and magnetic susceptibility with experimental results quantitatively~\cite{Yamaji2014}.
To utilize the parallel computing infrastructure with narrow bandwidth and distributed-memory
architectures, efficient, user-friendly, and highly parallelized diagonalization packages are highly desirable.

$\HPhi$, a flexible diagonalization package for solving quantum lattice hamiltonians,
has been developed to be such a descendant of the pioneering package TITPACK.
The Lanczos method for calculations of the ground state and a few excited states properties,
and finite temperature calculations based on thermal pure quantum states~\cite{Sugiura2012} are implemented in
the package $\HPhi$, with an easy-to-use and flexible user interface.
By using $\HPhi$, you can analyze a wide range of quantum lattice hamiltonians including
simple Hubbard and Heisenberg models, multi-band extensions of the Hubbard model,
exchange couplings that break SU(2) symmetry of quantum spins such as Dzyaloshinskii-Moriya
and Kitaev interactions, and Kondo lattice models describing itinerant electrons coupled with
quantum spins. $\HPhi$ calculates a variety of physical quantities such as internal energy at zero temperature or finite temperatures, temperature dependence of specific heat, charge/spin structure factors, and so on. A broad spectrum of users including experimental scientists is cordially welcome.

\section{}

\begin{eqnarray}
H = -\mu \sum_{i \sigma} c^\dagger_{i \sigma} c_{i \sigma} 
- \sum_{i \neq j \sigma} t_{i j} c^\dagger_{i \sigma} c_{j \sigma} 
+ \sum_{i} U n_{i \uparrow} n_{i \downarrow}
+ \sum_{i \neq j} V_{i j} n_{i} n_{j},
\end{eqnarray}

\begin{eqnarray}
H &=& -h \sum_{i} S_{i z} + \Gamma \sum_{i} S_{i x} + D \sum_{i} S_{i z} S_{i z}
\nonumber \\
&&+ \sum_{i j} \left( J_{i j x} S_{i x} S_{j x} + J_{i j y} S_{i y} S_{j y} + J_{i j z} S_{i z} S_{j z} 
\right),
\end{eqnarray}

\begin{eqnarray}
H = - \mu \sum_{i \sigma} c^\dagger_{i \sigma} c_{i \sigma} 
- t \sum_{\langle i j \rangle \sigma} c^\dagger_{i \sigma} c_{j \sigma} 
+ \frac{J}{2} \sum_{i} \left\{
S_{i}^{+} c_{i \downarrow}^\dagger c_{i \uparrow}
+ S_{i}^{-} c_{i \uparrow}^\dagger c_{i \downarrow}
+ S_{i z} (n_{i \uparrow} - n_{i \downarrow})
\right\},
\end{eqnarray}

\paragraph{Installation} If the document class \emph{elsarticle} is not available on your computer, you can download and install the system package \emph{texlive-publishers} (Linux) or install the \LaTeX\ package \emph{elsarticle} using the package manager of your \TeX\ installation, which is typically \TeX\ Live or Mik\TeX.

\paragraph{Usage} Once the package is properly installed, you can use the document class \emph{elsarticle} to create a manuscript. Please make sure that your manuscript follows the guidelines in the Guide for Authors of the relevant journal. It is not necessary to typeset your manuscript in exactly the same way as an article, unless you are submitting to a camera-ready copy (CRC) journal.

\paragraph{Functionality} The Elsevier article class is based on the standard article class and supports almost all of the functionality of that class. In addition, it features commands and options to format the
\begin{itemize}
\item document style
\item baselineskip
\item front matter
\item keywords and MSC codes
\item theorems, definitions and proofs
\item lables of enumerations
\item citation style and labeling.
\end{itemize}

\section{Front matter}

The author names and affiliations could be formatted in two ways:
\begin{enumerate}[(1)]
\item Group the authors per affiliation.
\item Use footnotes to indicate the affiliations.
\end{enumerate}
See the front matter of this document for examples. You are recommended to conform your choice to the journal you are submitting to.

\section{Bibliography styles}

There are various bibliography styles available. You can select the style of your choice in the preamble of this document. These styles are Elsevier styles based on standard styles like Harvard and Vancouver. Please use Bib\TeX\ to generate your bibliography and include DOIs whenever available.

Here are two sample references: \cite{Feynman1963118,Dirac1953888}.

\section*{References}

\bibliography{mybibfile}

\end{document}
